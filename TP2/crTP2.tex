\documentclass[a4paper,11pt]{exam}
\usepackage{style/styleCompteRendu}



\begin{document}

\Noms{Valentin Verstracte \\ Evan Petit}
\Titre{TP2 -  Systèmes de fichiers - commandes}


%/////////////////////////////////////////////////////////////////////////////////////////////////////////////////////
\exercice{Manipulation d'une arborescence de fichiers}
{
      \begin{enumerate}
	\item Déterminer votre numéro d’utilisateur (uid) sur le système ainsi que le nom et numéro du groupe auquel vous appartenez (gid).
\reponse{\commande{>echo \$UID\\>id -g v222843}}	
	  \item Déplacer vous dans votre répertoire d’accueil avec la commande \texttt{cd}. Utiliser la commande \texttt{pwd}.
\reponse{Retourne le chemin du répertoire courant}	  
	  \item Quels sont les droits d'accès de votre répertoire d’accueil? Cela vous parait-il logique?
\reponse{\commande{> ls -dl\\> drwx------ 68 vv224843...}\\ Commentaires : Oui cela semble logique que l'étudiant possède tous les droits dans son dossier personnel, et aucun dans les dossiers des autres.}	  
	  \item Télécharger le fichier Cyrano.txt
	    \begin{enumerate}
	    \item Afficher ce fichier par les commandes \texttt{cat} et \texttt{more} et tester dessus la commande \texttt{wc}.
\reponse{\commande{> wc -l cyrano.txt\\> 9}}	 	    
	    \item Chercher les lignes contenant le mot \texttt{monsieur} dans ce fichier avec la commande \texttt{grep}. Tester les différentes options.
\reponse{\commande{> grep "monsieur" cyrano.txt}}	 	    
	    \end{enumerate}
	  \item Tester sur le contenu d'un répertoire entier la commande \texttt{file} (pour avoir plusieurs résultats différents).
\reponse{\commande{> file ../TP2\\../TP2: directory}}	 	  
	\item Créer un dossier \texttt{MonDossier}. Copier puis déplacer des fichiers d'un autre répertoire dans ce dossier. Changer les droits de ce dossier. Supprimer un des fichiers contenus dans ce dossier. Supprimer ce dossier.
	\item Retrouver un fichier sur votre session à partir de la commande \texttt{find} et du nom du fichier.
\reponse{\commande{> find cyrano.txt\\cyrano.txt}}	 	
	\item Combien y a t-il sur votre session de fichiers avec l'extension \texttt{.pdf}, de plus de 20Mo?
\reponse{\commande{> find . -name '*.pdf' -size +20M}\\ Résultat : Ne retourne aucun fichier ( aucun pdf > 20Mo )}	
      \item Sur un fichier volumineux, comparer la taille du fichier et le volume utilisé sur le disque.
\reponse{\commande{> stat crTP2.pdf | grep "Taille"\\Taille : 58384         Blocs : 120}\\ Commentaires : Le fichier possède une taille de 58384 octets, cela représnete 120 blocs mémoire.}      
      \end{enumerate}
}


%/////////////////////////////////////////////////////////////////////////////////////////////////////////////////////
\exercice{Redirection}
{
\begin{enumerate}
 \item Créer à partir du shell un fichier texte qui contiendra le nom de la machine, la date et le répertoire courant, puis une ligne vide et le contenu du fichier \texttt{/etc/services}.
    \reponse{\commande{>( hostname ; date ; echo " " ; cat /etc/services ) > "titi.txt"}}
 \item Lancer la commande   \commande{cat fichierInconnu 2> fichRed}. Puis afficher le contenu de \texttt{fichRed}.
    \reponse{Résultats : fichRed contient  cat: fichierInconnu: Aucun fichier ou dossier de ce type\\
	      Commentaires : Le message d'erreur de la commande cat est envoyée dans le fichier fichRed}
\end{enumerate}

 	

}

%/////////////////////////////////////////////////////////////////////////////////////////////////////////////////////
\exercice{Les filtres}
{
	Reprenez le fichier \texttt{Cyrano.txt}. À partir des commande \texttt{tr}, \texttt{head}, \texttt{tail} et \texttt{sort}, donner les commandes pour ...
	\begin{enumerate}
	\item Afficher les lignes dans l'ordre croissant, dans l'ordre décroissant, dans un ordre aléatoire.
\reponse{\commande{> sort cyrano.txt\\> sort -r cyrano.txt\\> sort -R cyrano.txt }}	
	\item Créer un fichier avec les 5 premières lignes, les 5 dernières, les lignes 5 à 10.
\reponse{\commande{> head -n 5 cyrano.txt > fichier.txt \&\& tail -n 5 cyrano.txt >> fichier.txt \&\& head -n 10 cyrano.txt | tail -n 5 >> fichier.txt}}	
	\item Remplacer les \texttt{a} par des \texttt{i}.
\reponse{\commande{> tr 'a' 'i' <cyrano.txt }}	
	\end{enumerate}
}

%/////////////////////////////////////////////////////////////////////////////////////////////////////////////////////
\exercice{Les pipes}
{
	À l'aide d'un pipe, écrire des enchaînements de commandes pour 
	\begin{enumerate}	
	\item Faire afficher le résultat d'un \texttt{ls} avec les noms de fichier en majuscules.
\reponse{\commande{> ls | grep \^ㅤ[A-Z]}}	
	\item Faire afficher les chemins de la variable PATH séparés par des \textbf{-} et non par des \textbf{:}
\reponse{\commande{> echo \$PATH | tr ':' '-'}}	
	\item Recopier dans un fichier nommé \texttt{listeVar} la liste des variable d’environnement dont le nom contient au moins une fois le caractère \_.
\reponse{\commande{> printenv | grep .'\_' \>\> listVar }}	
	\item afficher l’ensemble des fichiers textes codés en UTF-8
\reponse{\commande{> file -i * | grep utf-8}}	
	\item  afficher l’ensemble des fichiers dans le répertoire REP que le propriétaire peut lire, modifier mais pas exécuter mais qu’un membre du groupe propriétaire peut juste lire.
\reponse{\commande{> ls -l REP | grep .-*r-rw-}}
	\end{enumerate}
}
 
\end{document}